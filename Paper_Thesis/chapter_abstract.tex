\chapter*{Abstract}
\addcontentsline{toc}{chapter}{Abstract}%

Modern photodetector applications encounter a large spectrum of different experimental environments. Prerequisites like working within strong magnetic fields, providing large intrinsic amplification, no dependence on high operation voltage, and space restrictions lead to the development of a new type of detectors, the semiconductor diodes. The \tit{Silicon Photomultiplier} (SiPM) integrates a huge number of \tit{avalanche photodiodes} as microcells within a small space. With a large intrinsic amplification of up to $10^6$, the SiPM comes in many different packaging sizes and works mainly with low voltage between $\SI{25}{\volt}$ and $\SI{85}{\volt}$. Its insensitivity to magnetic fields and single photon counting capability in combination with suited scintillators make SiPMs a valuable choice for modern challenging applications, i.e. \tit{nuclear magnetic resonance imaging} or \tit{positron emission tomography}. \par 
In this thesis, the first three chapters cover the theoretical foundation, i.e. particle interaction with matter, scintillators and semiconductor photodiodes. In the fourth chapter, the development, characterization, and application of a SiPM-based readout-module will be depicted. This includes temperature dependent measurements of IV-characteristics for determining the breakdown voltage and operation voltage with different configurations. In addition, the dark count rate and the dark electron spectrum are measured. The signal readout for raw and amplified solutions will be introduced. Furthermore, the fifth chapter will cover the measurement of efficiency and timing resolution properties in combination with a plastic scintillator, \tit{EJ-248M} from \tit{ELJEN}, in which the propagation time of photons in matter will be examined. In closing, the energy resolution for two prominent inorganic scintillators, LYSO and \pwo{}, will be tested by taking energy spectra with various calibration sources. The last part of this thesis will present a measurement of the energy loss of cosmic muons in lead tungstate.  

\chapter*{Zusammenfassung}
\addcontentsline{toc}{chapter}{Zusammenfassung}%

Weitreichende Anforderungen stellen Photodetektoren in modernen Anwendungen vor immer neue Herausforderungen. Einschr{\"a}nkungen wie das Funktionieren in starken magnetischen Feldern, eine gro\ss e intrinsische Verst{\"a}rkung, Niedrigspannungsversorgung und Handlichkeit f{\"u}rten zur Entwicklung einer neuen Art von Detektoren, den Halbleiterdioden. Der Siliziumphotovervielfacher (SiPM) integriert eine gro\ss e Menge an Lawinenphotodioden (APD) als Mikrozellen auf kleinem Raum. Mit einer gro\ss en intrinsischen Ver{\"a}rkung von bis zu $10^{6}$ werden SiPMs in vielen verschiedenen Gr{\"o}\ss en hergestellt und werden haupts{\"a}chlich mit Niedrigspannung, typisch zwischen $\SI{25}{\volt}$ und $\SI{85}{\volt}$, betrieben. Die Unempfindlichkeit gegen{\"u}ber magnetischen Feldern und die F{\"a}higkeit, einzelne Photonen z{\"a}hlen zu k{\"o}nnen, machen die SiPM in Kombination mit einem geeignetem Szintillator zu einer exzellenten Wahl, um in anspruchsvollen Anwendungen, wie zum Beispiel die Magnetresonanz-Thomographie oder die Positron-Emissions-Thomographie, eingesetzt zu werden. \par
In den ersten drei Kapiteln dieser Thesis werden die Grundlagen dieser Arbeit behandelt, die Wechselwirkung von Teilchen mit Materie, Szintillatoren und Halbleiter-Photodetektoren. Im vierte Kapitel wird die Entwicklung, Charakterisierung und Anwendung eines SiPM-basierten Auslesemoduls vorgestellt. Unter anderem wird die Tem\-pe\-ra\-tur\-ab\-h{\"a}n\-gig\-keit der Strom-Spannungscharakteristik f{\"u}r verschiedene Konfigurationen untersucht, um aus diesen die Durchbruchspannung und optimale Betriebsspannung zu erlangen. Zus{\"a}tzlich werden die Dun\-kel\-z{\"a}hl\-ra\-ten und Dunkelspektren gemessen, sowie die Auslese von Roh- und verst{\"a}rkten Signalen vorgestellt. Im f{\"u}nften kapitel wird die Effizienz und Zeitaufl{\"o}sung in Verbindung mit einem Plastikszintillator, \tit{EJ-248M} von \tit{ELJEN}, untersucht, indem die Laufzeiten von Photonem im Szintillatormaterial gemessen werden. Zus{\"a}tzlich wird die E\-ner\-gie\-auf\-l{\"o}\-sung in Verbindung mit zwei inorganischen Szintillatoren, LYSO und \pwo{} getestet. Als Abschluss wird das Energieverlust-Spektrum von kosmischen Myonen in Bleiwolframat aufgenommen.
















