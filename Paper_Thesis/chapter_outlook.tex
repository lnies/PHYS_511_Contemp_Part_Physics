\chapter{Summary and outlook}

The development and characterization of the SiPM-based readout-board was a major part of this thesis. Measuring the temperature dependency of the breakdown and operation voltage brought insight into the diverse behavior of the different boards.\par 
The single SiPM showed the expected behavior with exponentially increasing current beyond breakdown, whereas the parallel and hybrid configurations had even higher currents. The data provided by \tit{KETEK} for the temperature coefficient $k_{BD}$ of the SiPMs used were confirmed for all configurations and the coefficient $k_{OP}$ of the operation voltage was found to be slightly different between the single and hybrid/parallel configurations. This might be due to  differences in production conditions between the individual SiPMs and may be solved by characterizing each SiPM by means of the temperature coefficients and IV-characteristics. Matching similar diodes enables one to adjust breakdown and operation voltage precisely. \par 
This behavior was also found by taking the dark spectra. The unique discrete spectrum could only be found for a single SiPM, whereas multiple diodes only showed a blurred picture. By matching similar SiPMs, this could be improved, since knowledge about the thermal noise enables one to set the threshold right above a certain number of cells firing to get a good signal-to-noise ratio. The count rate for multiple SiPMs was found to be roughly $N$ times as high as the rate of a single SiPM, where $N$ is the number of diodes used. For a single SiPM, the slope of the count rate again showed the stepwise behavior, although multiple SiPMs did not. The slope for the hybrid configuration was found to be twice ($\sqrt{N}=2$) as high as the rate for a single or parallel SiPMs. \par 
By comparing the raw signals of the different configurations attached to the plastic scintillator \tit{EJ-248M}, it was discovered that the height of the single and parallel configurations are equal and twice as high as for the hybrid configuration. Again, there is a $\sqrt{N}$ dependence as proposed by \cite{sebastian}. Using the preamplifier showed that the hybrid configuration in fact has the shortest rise time compared with a single diode and the parallel configuration, which has the longest one. \par 
When it came to mechanically mounting the boards, one benefits by the chosen design of the board: due to its small size it was easily attached to the thin sides of the scintillator plate and was supported by the design of the spacing masks with reflective foil. Coupling to the crystals was simpler as anticipated because of various possibilities of arranging the SiPMs on the board and even daisy-chaining two or more boards electrically. \par 
One downside of this design is the connection between board and preamplifier: using a coaxial cable led to a huge amount of noise and ringing. This was avoided by directly connecting the boards via pin headers, which made the whole setup unhandy. Prospective versions of the SiPM-board are planned to integrate an on-board preamplifier. \par 
The efficiency measurements delivered an ambivalent result: even though the plastic scintillator plate and the SiPM-setup was designed symmetrically, an asymmetric result was yielded. Despite the count rate of both boards showed similar behavior, one detector was nearly twice as efficient in counting at certain positions, i.e. in front of its opposite detector. This circumstance was again observed in the timing resolution and propagation time measurement. The resolution ranged from $\SI{700}{\pico\second}$ up to $\SI{1300}{\pico\second}$. The results of the propagation time measurement suggests that the less effective detector should have been adjusted more carefully in terms of threshold and gain. Nevertheless, one of two detectors showed sufficient performance and by using more up-to-date processing electronics an even better time resolution might have been achieved. \par 
The energy measurement with the small LYSO crystal yielded some interesting energy spectra of the calibration sources and reached resolutions ranging from $20\%$ to $30\%$ depending on the source. Since cooling did not improve the energy resolution significantly, the conclusion has been drawn that the number and the size of the microcells might be a crucial factor. Shrinking the size and increasing the number of microcells should improve the sensitivity of the SiPM coupled to a bright scintillator. KETEK claims \cite{ketek_preamp} that a new series of SiPMs with a microcell size of $\SI{15}{\micro\meter}$ reaches a resolution of $13\%$ for the $\SI{511}{\keV}$ peak at a comparably small LYSO crystal. \par 
The daisy-chaining of two SiPM-boards created a $3\times 2$ matrix of parallel SiPMs to be attached to a small volume \pwo{} crystal. At room temperature, the energy spectra of the sources were barely distinguishable from the background and noise of the setup. When cooled, due to higher light yield and lower noise, several spectra could be taken and one was able to identify the photopeaks, yet the resolution was poor, as expected. For lightweak scintillators, SiPMs with larger microcells might be more suitable to choose since only few photons have to be detected and noise will be smaller. \par 
The muon measurements demonstrated that the \pwo{} shows a significant intrinsic background which was avoided by using a trigger system consisting of two plastic scintillators with photomultiplier tube readout. These precautions yielded a distinct energy loss spectrum of cosmic muons. 










 
