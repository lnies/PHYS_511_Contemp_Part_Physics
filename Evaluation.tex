\documentclass[12pt]{article}
\usepackage{a4}
\usepackage[english]{babel}
\setlength{\parindent}{0.35cm}
\pagestyle{headings}
\usepackage{graphicx}
\usepackage{grffile}
%Multiple picture in one figure
%\usepackage{subfigure}
\usepackage{subfig}
\usepackage[utf8]{inputenc}
\usepackage{listings}
\usepackage{color}
\usepackage{wrapfig}
%Floating-Umgebungen
\usepackage{float}
%Math-Environment
\usepackage{amsmath}
\usepackage{amssymb}
\usepackage{bbm}
%Better SI-Units
\usepackage{siunitx}
%Using Appendix
\usepackage[title]{appendix}
%Using URL
\usepackage[hidelinks]{hyperref}
%Using Colored Tables
\usepackage{colortbl}
\newcommand{\gray}{\rowcolor[gray]{.90}}
\usepackage{esvect}
% Use fancy tables
\usepackage{tabularx}
% Build fancy tables
\usepackage{booktabs}
\usepackage{soul}
% Configure enumeration
\usepackage{enumitem}
%Configure geometry
\usepackage[letterpaper]{geometry}
\geometry{
	letterpaper,
	left=3cm,
	right=3cm,
	top=3cm,
	bottom = 3cm,
	}

\lstset{
	language=C++,
	basicstyle=\small\ttfamily,
	keywordstyle=\color{blue}\ttfamily,
	stringstyle=\color{red}\ttfamily,
	commentstyle=\color{green}\ttfamily,
	morecomment=[l][\color{magenta}]{\#},
}


\usepackage{amsthm}

\renewcommand\qedsymbol{$\blacksquare$}
\newtheorem{theorem}{Theorem}[section]

\begin{document}
\noindent
\begin{center}
	\centering
	\Large{\textbf{Paper Evaluation and Summary}}
\end{center}
\textbf{\underline{Name}}: Lukas Nies \\
\noindent
\textbf{\underline{Paper}}: Baker et al. (Limit on EDM measured at ILL, 2006) \\[0.5cm] 
\noindent
\begin{itemize}
	\item \ul{Motivation}
	\noindent
	Measuring the electric dipole moment of neutrons is a test of Physics beyond the standard model and can, if no EDM is found, constrain theories about the baryon asymmetry in the universe and Supersymmertry.
	\item \ul{What is the main finding of this paper and why is it important?} \\
	\noindent
	The collaboration measures an upper limit for the EDM in the order of $\si{10\tothe{-26}\elementarycharge\centi\meter}$ which is the most precise measurement at this point.

	\item \ul{Describe at a high level the basic technique used. Try a series of "steps" here if necessary, if there is a sequence to be followed (like a recipe).} \\
	\noindent
	Batch cycle:
	\begin{itemize}
		\item Ultra cold neutrons (UCNs) are produced 
		\item UCNs get spin polarized by a magnetized iron foil
		\item Then enter 21 liter cylindrical trap with external vertical magnetic field of $\SI{1}{\micro\tesla}$
		\item One fill period is $\SI{20}{\second}$ then the door gets closed
		\item Two electrodes generate electric field of $\SI{100}{\kilo\volt\per\centi\meter}$
		\item Measurement of transition frequency with "Ramsey separated-oscillatory-field magnetic oscillatory method":
		\begin{itemize}
			\item All neutrons start spin-aligned with external magnetic field
			\item "$\pi/2$ perturbation pulse" (small transverse magnetic field) flips spin with $90^{\circ}$, now free precession in horizontal plane
			\item After some time second pulse is sent to rotate the spins again: If the two clocks (external pulse and neutron precession) are in phase then spin will be completely flipped. If not a phase difference accumulates over the free precession time and the spin is either flipped back upwards or downwards.
		\end{itemize}
		\item After Ramsey sequence was applied, door opens and neutrons fall down on the polarization foil
		\item Only neutrons with same spin as before can leave, get counted in gaseous $^{3}He$ counter
		\item Other neutrons get rejected, spin-flipped, and then counted
	\end{itemize}

	\item \ul{Choose an interesting technical aspect of the experiment and describe its relation and importance to the measurement.} \\
	\noindent
	The experiment is equipped with a $^{199}Hg$ magnetometer consisting of mercury atoms injected in the trap to measure magnetic field drifts within the chamber. A $\pi/2$ pulse rotates the spins in a the horizontal plane where they precess freely. Measuring the absorption of circular polarized light gives a measure of the average strength of the magnetic field. Precise knowledge of the magnetic field is crucial for the measurement
	
	\item \ul{Pick one systematic uncertainty issue that you find interesting and describe its importance and the author's method of addressing it.}\\
	\noindent
	In dealing with the geometric phase shift, the collaboration also had to take the earths rotation into account (!) since the atoms and neutrons trapped in the vacuum process freely while the earth rotates beneath them. 
	
	\item \ul{Where did you get lost? Was there anything you did not understand?:} \\
	\noindent
	\begin{enumerate}
		\item A hard paper to read, only possible to roughly understand with "explanations" by Harris, who's paper is very well written.
		\item How does the polarization foil work?
		\item How did they measure that the COM of the neutrons is $\SI{2.8}{\milli\meter}$ below the mercury atoms?
		\item How is the EDM extracted?
		\item Geometric phase effect...
	\end{enumerate}
	
\end{itemize}


\end{document}  