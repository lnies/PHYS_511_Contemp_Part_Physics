\documentclass[12pt]{article}
\usepackage{a4}
\usepackage[english]{babel}
\setlength{\parindent}{0.35cm}
\pagestyle{headings}
\usepackage{graphicx}
\usepackage{grffile}
%Multiple picture in one figure
%\usepackage{subfigure}
\usepackage{subfig}
\usepackage[utf8]{inputenc}
\usepackage{listings}
\usepackage{color}
\usepackage{wrapfig}
%Floating-Umgebungen
\usepackage{float}
%Math-Environment
\usepackage{amsmath}
\usepackage{amssymb}
\usepackage{bbm}
%Better SI-Units
\usepackage{siunitx}
%Using Appendix
\usepackage[title]{appendix}
%Using URL
\usepackage[hidelinks]{hyperref}
%Using Colored Tables
\usepackage{colortbl}
\newcommand{\gray}{\rowcolor[gray]{.90}}
\usepackage{esvect}
% Use fancy tables
\usepackage{tabularx}
% Build fancy tables
\usepackage{booktabs}
\usepackage{soul}
% Configure enumeration
\usepackage{enumitem}
%Configure geometry
\usepackage[letterpaper]{geometry}
\geometry{
	letterpaper,
	left=3cm,
	right=3cm,
	top=3cm,
	bottom = 3cm,
	}

\lstset{
	language=C++,
	basicstyle=\small\ttfamily,
	keywordstyle=\color{blue}\ttfamily,
	stringstyle=\color{red}\ttfamily,
	commentstyle=\color{green}\ttfamily,
	morecomment=[l][\color{magenta}]{\#},
}


\usepackage{amsthm}

\renewcommand\qedsymbol{$\blacksquare$}
\newtheorem{theorem}{Theorem}[section]

\begin{document}
\noindent
\begin{center}
	\centering
	\Large{\textbf{Paper Evaluation and Summary}}
\end{center}
\textbf{\underline{Name}}: Lukas Nies \\
\noindent
\textbf{\underline{Paper}}: Graner et al. (Limit on AEDM measured at UW, 2016) \\[0.5cm] 
\noindent
\begin{itemize}
	\item \ul{Motivation} \\
	\noindent
	Measuring the electric dipole moment of the $^{199}Hg$ atom is a test of Physics beyond the standard model and can, if no AEDM is found, constrain theories about the baryon asymmetry in the universe and Supersymmertry. If an AEDM is found, this could be evidence for a violation of CP in strong interactions and may as well could explain today's excess of matter.
	\item \ul{What is the main finding of this paper and why is it important?} \\
	\noindent
	The collaboration measures an upper limit for the atomic EDM in the order of $\si{10\tothe{-30}\elementarycharge\centi\meter}$ which is the most precise measurement at this point.

	\item \ul{Describe at a high level the basic technique used. Try a series of "steps" here if necessary, if there is a sequence to be followed (like a recipe).} \\
	\noindent
	Four stacked vapor cells made of fused silica are filled with a small amount of CO buffer gas and $^{199}Hg$ and are placed in an external magnetic field. The atoms are transverse polarized and precess. An electric field with $\SI{10}{\kilo\volt}$ is applied for the inner two cells. The two outer cells don't have an electric field and are used as a magnetometer. One pump-probe cycle consists of:
	\begin{itemize}
		\item Pump period ($\SI{30}{\second}$): Utilizing circularly polarized laser light to coherently polarize atoms 
		\item Equilibration period ($\SI{20}{\second}$): 
		\item Initial probe period A ($\SI{20}{\second}$): Probe atoms with linear polarized, detuned, and attenuated laser light 
		\item Free precession period ($\SI{170}{\second}$): Dark period
		\item Final probe period B ($\SI{30}{\second}$): Probe atoms again
	\end{itemize} 
	The output beams are rotated and separated into the p and s components, each component is measured with an enhanced UV-photodiode.    
	
	\item \ul{Choose an interesting technical aspect of the experiment and describe its relation and importance to the measurement.} \\
	\noindent
	The experiment is built such that very small disturbances in the magnetic local field can be measured. For this, $^{199}Hg$ atoms in the two outer cells are used as magnetometers. Knowing the exact magnetic field strength in the vapor chambers is crucial to get high precision and to avoid systematical error.  
	
	\item \ul{Pick one systematic uncertainty issue that you find interesting and describe its importance and the author's method of addressing it.}\\
	\noindent
	Leakage current flowing around the vapor cell increase the magnetic field producing a comparable Larmor shift as the AEDM would do. To monitor this, high sensitive ampere meters where connected to the chamber. An upper limit of flown current of $\SI{40}{\femto\ampere}$ was set. 
	
	\item \ul{Where did you get lost? Was there anything you did not understand?:} \\
	\noindent
	\begin{itemize}
		\item Rotating copper plate for syncing laser with Larmor frequency?
		\item Why is probing laser detuned and attenuated?
		\item Potential feedthrough of the cell motion
	\end{itemize}
	
	
	
\end{itemize}


\end{document}  