\documentclass[12pt]{article}
\usepackage{a4}
\usepackage[english]{babel}
\setlength{\parindent}{0.35cm}
\pagestyle{headings}
\usepackage{graphicx}
\usepackage{grffile}
%Multiple picture in one figure
%\usepackage{subfigure}
\usepackage{subfig}
\usepackage[utf8]{inputenc}
\usepackage{listings}
\usepackage{color}
\usepackage{wrapfig}
%Floating-Umgebungen
\usepackage{float}
%Math-Environment
\usepackage{amsmath}
\usepackage{amssymb}
\usepackage{bbm}
%Better SI-Units
\usepackage{siunitx}
%Using Appendix
\usepackage[title]{appendix}
%Using URL
\usepackage[hidelinks]{hyperref}
%Using Colored Tables
\usepackage{colortbl}
\newcommand{\gray}{\rowcolor[gray]{.90}}
\usepackage{esvect}
% Use fancy tables
\usepackage{tabularx}
% Build fancy tables
\usepackage{booktabs}
\usepackage{soul}
% Configure enumeration
\usepackage{enumitem}
%Configure geometry
\usepackage[letterpaper]{geometry}
\geometry{
	letterpaper,
	left=3cm,
	right=3cm,
	top=3cm,
	bottom = 3cm,
	}

\lstset{
	language=C++,
	basicstyle=\small\ttfamily,
	keywordstyle=\color{blue}\ttfamily,
	stringstyle=\color{red}\ttfamily,
	commentstyle=\color{green}\ttfamily,
	morecomment=[l][\color{magenta}]{\#},
}


\usepackage{amsthm}

\renewcommand\qedsymbol{$\blacksquare$}
\newtheorem{theorem}{Theorem}[section]

\begin{document}
\noindent
\begin{center}
	\centering
	\Large{\textbf{Paper Evaluation and Summary}}
\end{center}
\textbf{\underline{Name}}: Lukas Nies \\
\noindent
\textbf{\underline{Paper}}: Nico et al. (Measurement of neutron lifetime using beam method) \\[0.5cm]
\noindent
\begin{itemize}
	\item \ul{What is the main finding of this paper and why is it important?} \\
	\noindent
	The mean neutron lifetime was found to be $\tau_n=886.3\pm 1.2\text{(stat)}\pm 3.2\text{(sys)}s$ and therefore is the most precise measurement utilizing the beam method. It still disagrees with the magnetic trap method. 
	\item \ul{Describe at a high level the basic technique used. Try a series of "steps" here if necessary, if there is a sequence to be followed (like a recipe).} \\
	\noindent
	\begin{enumerate}
		\item Generate cold neutrons, guide into the proton trap 
		\item Start the trapping cycle:
		\begin{itemize}
			\item Trapping: trap protons produced by neutron decay for $10ms$.
			\item Counting: Door gets grounded and a ramped potential "flushes" out the produced protons which are guided to the silicon detector (enabled for $21\mu s$).
			\item Clearing: ramp potential gets maintained for another $33\mu s$ to clear the trap from all charged particles
		\end{itemize}
		\item Measure resulting protons with silicon surface barrier detector (must be aligned precisely with proton beam).
	\end{enumerate}
	\item \ul{Choose an interesting technical aspect of the experiment and describe its relation and importance to the measurement.} \\
	\noindent
	The alignment of the proton detector to measure the proton beam is quite sophisticated to avoid loss of counts due to exceeding the active area of the silicon. To do so, optical measurements were performed first to get a "rough" alignment. Furthermore, an electron beam from a $^{210}$Pb-$^{210}$Bi-$^{210}$Po source was used to monitor the the beam position on the detector. Last, the same procedure was performed with decay protons. All measurements agreed, according to the paper, better than $1 mm$.    
	\item \ul{Pick one systematic uncertainty issue that you find interesting and describe  its importance and the author's method of addressing it.}\\
	\noindent
	
	\item \ul{Fascinating aspect:} \\
	\noindent
	The experiment is described much more sophisticated than the Pattie et al. paper. The method also necessitates the application of much more systematic corrections to the data then the magnetic trap method. 
\end{itemize}


\end{document}  